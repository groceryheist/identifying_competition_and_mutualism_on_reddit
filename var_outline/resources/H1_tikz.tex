\def\seawacircle{(0cm,-7cm) circle (1.5cm)}
\def\seacircle{(-0.8cm,-8cm) circle (1.5cm)}
\def\seabikecircle{(2cm,-7.2cm) circle (1.5cm)}
\def\seaentcircle{(1.5cm,-9cm) circle (1.5cm)}

\tikzstyle{frame_black} = [line width=1pt, draw=black,inner sep=0.8em,minimum width=\textwidth,minimum height=3cm]
\begin{tikzpicture}  [->,>=stealth',shorten >=1pt,auto,
      thick,
      obs/.style={rectangle,draw,font=\sffamily\small,inner sep=0.5em},
      latent/.style={rectangle,draw,font=\sffamily\small,fill=gray!20, inner sep=0.5em},
      label/.style={font=\scshape\small},
      obsmic node/.style={ellipse,draw,font=\sffamily\small},
      unobsmic node/.style={ellipse,draw,font=\sffamily\small,fill=gray!20},
      declare function={rr=1.25+0.2*rnd;}]
      \draw plot[smooth cycle, variable=\t, samples at={0,45,...,315}] (\t:2*rr);
% put them inside a shape that looks like a soup
      \node[] (hidden1) [yshift=2.5cm]{};
      \node[obs] (seawa1) [yshift=1cm,xshift=-0.5cm] {/r/seattlewa};
      \node[obs] (sea1) [right of=seawa1, xshift=1cm, yshift=-0.10cm] {/r/seattle};
      \node[obs] (seabike1) [below left of=seawa1, yshift=-0.8cm] {/r/seattlebike};
      \node[obs] (seaent1) [below left of=sea1, yshift=-1cm] {/r/seattleent};

      \node[label] (lab1) [yshift=-3.5cm] {A ``pure'' population ecology. All groups share a niche.};
      \node[frame_black,fit=(seawa1)(hidden1)(sea1)(seabike1)(seaent1)(lab1)] (frame1) {};

      \draw plot[smooth cycle, variable=\t, samples at={0,45,...,315},yshift=-5] (\t:2*rr);
      \node (hidden2) [yshift=-5.5cm] {};
      

      \begin{scope}
        \clip \seaentcircle;
        \fill[gray!20] \seawacircle;
        \fill[gray!20] \seabikecircle;
        \fill[gray!20] \seacircle;
      \end{scope}

      \draw \seawacircle node (seawa2) [below] {/r/seattlewa};
      \draw \seacircle node (sea2) [below] {/r/seattle};
      \draw \seabikecircle node (seabike2) [below] {/r/seattlebike};
      \draw \seaentcircle node (seaent2) [below] {/r/seattleent};
      \node[label] (lab2) [yshift=-11.2cm,align=center] {Zhu et al model overlapping niches. \\ Niche overlaps with /r/seatleent's are shaded.};

      \node[frame_black,fit=(seawa2)(hidden2)(sea2)(seabike2)(seaent2)(lab2)] (frame2) {};

      \node[obs] (seattle) at (1.5,-16.5) {/r/seattle};
      \node[obs] (seattlewa) at (1.5,-13.5) {/r/seattlewa};
      \node[obs] (seattlebike) at (-1.5,-16.5) {/r/seattlebike};
      \node[obs] (seattleents) at (-1.5,-13.5) {/r/seattleents};

      \path[every node/.style={font=\sffamily}]
      (seattle) edge[bend left=15] (seattlewa)
      (seattle) edge[] (seattlebike)
      (seattle) edge[] (seattleents)
      (seattlewa) edge[bend left=15] (seattle)
      (seattlewa) edge[] (seattlebike)
      (seattlewa) edge[] (seattleents)
      (seattlebike) edge[] (seattleents)
      (seattlebike) edge[bend left=15] (seattle)
      (seattlebike) edge[bend left=15] (seattlewa)
      (seattleents) edge[bend left=15] (seattlewa)
      (seattleents) edge[bend left=15] (seattlebike)
      (seattleents) edge[bend left=15] (seattle);

      \node[label] (lab3) [yshift=-17.8cm,align=center] {Community ecology models a network of relations.};
      \node[frame_black,fit=(lab3)(seattlewa)(seattle)(seattlebike)(seattleents)] (frame3) {};

\end{tikzpicture}

%%% Local Variables:
%%% mode: latex
%%% TeX-master: t
%%% End:

%%% Local Variables:
%%% mode: latex
%%% TeX-master: t
%%% End:
